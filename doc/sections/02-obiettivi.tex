\section{Obiettivi dello studio}\label{sec:obiettivi}
L'obiettivo dello studio è quello di minimizzare il numero degli sportelli attivi al fine di garantire il soddisfacimento di differenti requisiti di qualità per ciascuna tipologia di servizio illustrata nella presentazione (sez. \ref{sec:presentazione}):

\begin{enumerate}[label=QoS-\arabic*), align=left, leftmargin=*]
\item I titolari di conto \textsl{BancoPosta} devono sperimentare un tempo d'attesa non superiore a tre volte il tempo medio di servizio generale dei clienti.
\item I clienti non titolari del conto \textsl{BancoPosta}:
\begin{enumerate}
\item Possessori di ticket di tipo \uo{} devono sperimentare un tempo d'attesa non superiore a sette volte il tempo medio di servizio dei clienti della medesima tipologia.
\item Possessori di ticket di tipo \pp{} devono sperimentare un tempo d'attesa non superiore a dieci volte il tempo medio di servizio dei clienti della medesima tipologia.
\item Possessori di ticket di tipo \sr{} devono sperimentare un tempo d'attesa non superiore a cinque volte il tempo medio di servizio dei clienti della medesima tipologia.
\end{enumerate}
\end{enumerate}

È opportuno specificare che nel momento in cui non si soddisfi il QoS per un cliente, quest'ultimo può decidere di abbandonare l'ufficio postale.
