\section{Modello Delle Specifiche}\label{sec:modello-specfiche}
Per lo studio in questione sono stati utilizzati dati provenienti da \textsl{"Principali dati economici e finanziari di Poste Italiane"}\footnote{\url{https://www.posteitaliane.it/it/performance-finanziaria.html}}, per comodità riportati in tabella \ref{table:modello-specifiche-1}.

\begin{table}[ht]
\centering
{\tablecolors
\begin{tabular}{| l | r |}
\hline
Grandezza & Valore misurato nel 2019 \\
\hline
N$^o$ di clienti al giorno & 1.4 mln \\
\hline
N$^o$ di uffici postali & 12 809 \\
\hline
\end{tabular}}
\scaption{Estratto dei dati di interesse a partire dalla fonte citata}
\label{table:modello-specifiche-1}
\end{table}

Di seguito sono riportate alcune assunzioni che saranno alla base di questa e delle successive fasi dello studio:
\begin{itemize}
\item I clienti arrivano all'ufficio postale ad istanti di tempo casuali, il che implica:
\begin{itemize}
\item Distribuzione poissoniana degli arrivi.
\item Distribuzione esponenziale dei tempi di inter-arrivo.
\end{itemize}
\item La probabilità che un cliente sia titolare di un conto \textsl{BancoPosta} è pari a $p_{BP} = 0.25$.
\item Le probabilità con cui ciascuna tipologia di ticket viene acquisita sono le seguenti:
\begin{equation*}
\begin{array}{l c l}
\uo{} & \rightarrow & p_{UO} = 0.5 \\
\pp{} & \rightarrow & p_{PP} = 0.35 \\
\sr{} & \rightarrow & p_{SR} = 0.15
\end{array}
\end{equation*} 
\item I tempi di servizio sono distribuiti esponenzialmente.
\item I clienti afferenti ad una stessa coda vengono serviti in accordo ad una disciplina FIFO (First-In, First-Out).
\item Il servizio di un cliente non può essere interrotto per favorire l'avanzamento di un altro con priorità superiore.
\end{itemize}

La politica di scheduling del sistema assume caratteristiche tipiche sia dello scheduling size-based che di quello astratto. In particolare:
\begin{itemize}
\item Ha caratteristiche astratte perché la probabilità di ricadere in una delle classi di priorità è funzione della frequenza con cui l'utente prende quella tipologia di ticket e non del tempo necessario all'elaborazione della richiesta da parte del servente. Infatti, il tempo necessario ad elaborare un ticket della classe $i$ non è necessariamente compreso in un intervallo $(x_{i-1}, x_i]$ bensì può assumere valori su tutto il semiasse reale positivo. 
\item Ha caratteristiche size-based perché classi di priorità diverse non condividono lo stesso tempo di servizio medio, ovvero $\exists\ i,j\ t.c.\ E[S_i] \neq E[S_j]$.
\end{itemize}

Assumendo che:
\begin{itemize}
\item Il tempo medio di servizio $E[S]$ sia pari a $10\ min$.
\item Il tempo medio di servizio per una richiesta di tipo \pp{} sia pari ad una volta e mezzo quello di una \uo{}.
\item Il tempo medio di servizio per una richiesta di tipo \sr{} sia pari a due volte quello di una \uo{}.
\end{itemize}
è possibile ricavare gli $E[S_k]$ come segue:
\begin{equation}
\begin{split}
E[S] &= p_{UO}\cdot E[S_{UO}] + p_{PP}\cdot E[S_{PP}] + p_{SR}\cdot E[S_{SR}] \\
&= p_{UO}\cdot E[S_{UO}] + \frac{3}{2}\cdot p_{PP}\cdot E[S_{UO}] + 2\cdot p_{SR}\cdot E[S_{UO}] \\
&= (p_{UO} + \frac{3}{2}\cdot p_{PP} + 2\cdot p_{SR}) \cdot E[S_{UO}]
\end{split}
\end{equation}
da cui:
\begin{equation}
\begin{cases}
E[S_{UO}] = \frac{E[S]}{(p_{UO} + \frac{3}{2}\cdot p_{PP} + 2\cdot p_{SR})} = \frac{400}{53} \simeq 7.5471698\ min \\[1em]
E[S_{PP}] = \frac{3}{2}\cdot E[S_{UO}] = \frac{600}{53} \simeq 11.3207547\ min \\[1em]
E[S_{SR}] = 2 \cdot E[S_{UO}] = \frac{800}{53} \simeq 15.0943396\ min
\end{cases}
\end{equation}

I risultati ottenuti possono essere verificati con il seguente consistency check:
\begin{equation}
p_{UO}\cdot E[S_{UO}] + p_{PP}\cdot E[S_{PP}] + p_{SR}\cdot E[S_{SR}] = 10\ min = E[S]\qquad (\mathbf{OK})
\end{equation}

Infine, a partire dalla seconda e dalla terza riga, è possibile ricavare il throughput del sistema $X$ come segue:
\begin{equation}
X = \frac{\# \lbrace \text{clienti al giorno} \rbrace}{\# \lbrace \text{uffici postali} \rbrace} \cdot \frac{1}{24 \cdot 60} \simeq \frac{109.2981497}{24 \cdot 60} \simeq 0.0759015\ req/min
\end{equation} 
















