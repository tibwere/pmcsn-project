\documentclass{article}

% Package necessari
\usepackage[a4paper]{geometry}
\usepackage[utf8]{inputenc}
\usepackage[italian]{babel}
\usepackage[T1]{fontenc}
\usepackage{amsmath}
\usepackage{amssymb}
\usepackage{graphicx}
\usepackage[table, dvipsnames]{xcolor}
\usepackage{listings}
\usepackage{hyperref}
\usepackage{enumitem}
\usepackage{fancyhdr}
\usepackage{algorithm}
\usepackage[noend]{algpseudocode}

% Impostazione delle lunghezze di alcuni elementi del documento
\setlength{\parskip}{1em}
\setlength{\parindent}{0em}
\setlength{\arrayrulewidth}{0.1em}

% Informazioni per la title page
\title{\small{Corso di Performance Modeling of Computer Systems \& Networks} \\
\huge{Studio delle prestazioni di un Ufficio Postale di \\
\textbf{Poste Italiane}}}

\date{A.A. 2020/2021}

\author{A. Chillotti\thanks{\texttt{\href{mailto:alessandro.chillotti@outlook.it}{alessandro.chillotti@outlook.it}}}
\and 
C. Cuffaro\thanks{\texttt{\href{mailto:cristiano.cuffaro@outlook.com}{cristiano.cuffaro@outlook.com}}} 
\and 
S. Tiberi\thanks{\texttt{\href{mailto:simone.tiberi.98@gmail.com}{simone.tiberi.98@gmail.com}}}
}

% Impostazione del package hyperref
\hypersetup{
    colorlinks=true,
    linktocpage=true,
    linkcolor=blue,
    urlcolor=blue,
    pdftitle={Studio delle prestazioni di un Ufficio Postale di Poste Italiane},
    pdfauthor={A. Chillotti, C. Cuffaro e S. Tiberi},
}

% Colori per i listing
\definecolor{code_red}{rgb}{0.6,0,0} % strings
\definecolor{code_green}{rgb}{0.25,0.5,0.35} % comments
\definecolor{code_purple}{rgb}{0.5,0,0.35} % keywords
\definecolor{code_background}{rgb}{0.95,0.95,0.92} % background

% Altri colori
\definecolor{forestgreen}{rgb}{0.13, 0.55, 0.13}
\definecolor{airforceblue}{rgb}{0.36, 0.54, 0.66}
 
% Stile del codice standard (C)
\lstset{
	language=C, 
	backgroundcolor=\color{code_background},
	frame=single,
	basicstyle=\ttfamily,
	keywordstyle=\color{code_purple}\bfseries,
	stringstyle=\color{code_red},
	commentstyle=\color{code_green},
	numbers=left,
	numberstyle=\small\color{gray},
	numbersep=5pt,
	tabsize=4,
	showtabs=false,
	showspaces=false,
	showstringspaces=false,
	escapechar=|, 
	captionpos=b,
	breaklines=true,
}

\pagestyle{fancy}
\fancyhf{}
\lhead{\small A.Chillotti, C. Cuffaro, S. Tiberi}
\rhead{\small Performance Modeling of Computer Systems \& Networks}
\cfoot{\thepage}
%\cfoot{Pagina \thepage}

% Spaziatura tabelle
\renewcommand{\arraystretch}{1.5}

\graphicspath{ {./figs/} }
% Definizione del colore delle tabelle
\newcommand{\tablecolors}[1][2]{\rowcolors{#1}{yellow!50}{yellow!25}}

% Definizione dello stile da usare per la P di probabilità (grassetto in math-mode)
\newcommand{\pr}{\mathbf{P}}

% Forzatura del displaystyle in math-mode
\everymath\expandafter{\the\everymath\displaystyle}

\newcommand{\scaption}[1]{\caption{\small{#1}}}
\newcommand{\ded}{{\color{red}\hat{k}}}

\newcommand{\uo}{\textsl{Unica Operazione}}
\newcommand{\pp}{\textsl{Pagamenti \& Prelievi}}
\newcommand{\sr}{\textsl{Spedizioni \& Ritiri}}

\begin{document}
\maketitle

\section{Presentazione del caso di studio}
Il sistema oggetto dell'analisi in questione eroga le seguenti tipologie di servizi:
\begin{enumerate}
\item \textbf{Unica Operazione} (e.g. ricarica \textsl{PostePay}, invio raccomandata e pagamento di massimo 3 bollettini)
\item \textbf{Pagamenti \& Prelievi} (e.g. pagamento di un numero arbitrario di bollettini, bollo auto e libretti)  
\item \textbf{Spedizioni \& Ritiri} (e.g. invio corrispondenza, lettere, pacchi e raccomandate)
\end{enumerate}

Per essere serviti i clienti possono:
\begin{itemize}
\item Recarsi all'ufficio postale, prendere un ticket relativo al servizio a cui sono interessati e mettersi in coda in attesa del proprio turno. Nel caso in cui essi dimostrano di essere titolari di un conto \textsl{BancoPosta} potranno accodarsi in una fila dedicata.
\item Prenotare un ticket mediante l'applicazione \textsl{"Ufficio Postale"} per una determinata fascia oraria, al fine di essere serviti dal primo sportello disponibile entro 40 minuti, ma non prima, dall'orario di prenotazione.
\end{itemize}

Un insieme di sportelli serve le richieste degli utenti in accordo alle seguenti regole: 
\begin{enumerate}[label=R\arabic*)]
\item I clienti titolari di un conto \textsl{BancoPosta} vengono serviti con una priorità maggiore rispetto agli altri, indipendentemente dal ticket scelto.
\item Poiché, per definizione, ticket di tipo \textbf{Unica Operazione} dovrebbero richiedere meno tempo per essere processati, viene assegnata loro la massima priorità
\item I ticket di tipo \textbf{Spedizioni \& Ritiri} vengono serviti da uno sportello dedicato il quale, in assenza di questa tipologia di ticket, opera come gli altri. Il comportamento di tale servente è schematizzato in figura \ref{fig:presentazione-1}. 
\end{enumerate}

\begin{figure}[ht]
\centering
\includegraphics[width=0.75\linewidth]{presentazione-1}
\scaption{Schema del comportamento del servente dedicato ai ticket di tipo \textbf{Spedizioni \& Ritiri}}
\label{fig:presentazione-1}
\end{figure}
\chapter{Obiettivi dello studio}\label{chp:obiettivi}
L'obiettivo dello studio è quello di minimizzare il numero degli sportelli operativi in un'intera giornata lavorativa, al fine di garantire il soddisfacimento di differenti requisiti di qualità per ciascuna tipologia di servizio illustrata nella presentazione del caso di studio (cap. \ref{chp:presentazione}):

\begin{itemize}
\item I clienti titolari di un conto \textsl{BancoPosta}:
\begin{enumerate}[label=QoS-\arabic*), align=left, leftmargin=*]
\item Possessori di ticket di tipo \uo{}, devono sperimentare un tempo d'attesa non superiore a $15\ min$.
\item Possessori di ticket di tipo \pp{}, devono sperimentare un tempo d'attesa non superiore a $20\ min$.
\item Possessori di ticket di tipo \sr{}, devono sperimentare un tempo d'attesa non superiore a $20\ min$.
\end{enumerate}
\item I clienti \textbf{non} titolari di un conto \textsl{BancoPosta}:
\begin{enumerate}[label=QoS-\arabic*), align=left, leftmargin=*]
\setcounter{enumi}{3}
\item Possessori di ticket di tipo \uo{}, devono sperimentare un tempo d'attesa non superiore a $30\ min$.
\item Possessori di ticket di tipo \pp{}, devono sperimentare un tempo d'attesa non superiore a $90\ min$.
\item Possessori di ticket di tipo \sr{}, devono sperimentare un tempo d'attesa non superiore a $45\ min$.
\end{enumerate}
\end{itemize}
\section{Modello Concettuale}\label{sec:modello-concettuale}
\begin{figure}[ht]
\centering
\includegraphics[width=\linewidth]{modello-concettuale-1}
\scaption{Diagramma del sistema \textbf{Poste Italiane}}
\label{fig:modello-concettuale-1}
\end{figure}

Il funzionamento del sistema è illustrato dal diagramma in figura \ref{fig:modello-concettuale-1}. Di seguito è riportata una descrizione degli elementi in esso utilizzati:
\begin{itemize}
\item Il rombo rappresenta un meccanismo di ripartizione del flusso in ingresso nelle opportune code, a seconda della titolarità o meno di un conto \textsl{BancoPosta} da parte dei clienti.
\item Ciascuna coda modella una fila di clienti possessori dello stesso tipo di ticket.
\item Ciascun servente rappresenta uno sportello dell'ufficio postale
\begin{itemize}
\item Il $k$-esimo servente (evidenziato in {\color{red} rosso}) rappresenta lo sportello dedicato per la gestione dei ticket di tipo \sr{}, il cui comportamento è stato già illustrato nel flow chart in figura \ref{fig:presentazione-1}.
\end{itemize}
\end{itemize}

Ad ogni istante di tempo, lo stato del sistema è univocamente determinato dai valori assunti dalle seguenti variabili di stato:
\begin{itemize}
\item Per ciascun servente $i$ (con $i \in \lbrace 1, 2, \dots, k-1, k+1, \dots, M \rbrace$) si ha:
\begin{equation}
Sportello_i \in \lbrace \mathtt{IDLE},\ \mathtt{BUSY} \rbrace 
\end{equation}
\item Per il $k$-esimo servente, ovvero quello dedicato, si ha:
\begin{equation}
Sportello_k \in \lbrace \mathtt{IDLE},\ \mathtt{BUSY\_GENERAL},\ \mathtt{BUSY\_SPED\_RIT}\rbrace 
\end{equation}
dove:
\begin{itemize}
\item \texttt{BUSY\_GENERAL} rappresenta il caso in cui il servente stia elaborando un ticket \textbf{non} di tipo \sr{}.
\item \texttt{BUSY\_SPED\_RIT} rappresenta il caso in cui il servente stia elaborando un ticket di tipo \sr{}.
\end{itemize}
\item Per ciascuna coda $(t, j)$ con:
\begin{itemize}
\item $t \in \lbrace \mathtt{BANCO\_POSTA},\ \mathtt{STANDARD}\rbrace$
\item $j \in \lbrace \mathtt{UNICA\_OP},\ \mathtt{PAGAM\_PREL},\ \mathtt{SPED\_RIT} \rbrace$
\end{itemize}
il numero di clienti in essa presenti è modellato da $Coda_{t,j} \in \mathbb{N}$.
\end{itemize}

\section{Modello Delle Specifiche}\label{sec:modello-specfiche}
Di seguito sono riportate alcune assunzioni che saranno alla base di questa e delle successive fasi dello studio:
\begin{itemize}
\item I clienti arrivano all'ufficio postale ad istanti di tempo casuali, il che implica:
\begin{itemize}
\item Distribuzione poissoniana degli arrivi.
\item Distribuzione esponenziale dei tempi di interarrivo.
\end{itemize}
\item La probabilità che un cliente sia titolare di un conto \textsl{BancoPosta} è pari a $p_{BP} = 0.25$.
\item Le probabilità con cui ciascuna tipologia di ticket viene acquisita sono le seguenti:
\begin{equation*}
\begin{array}{l c l}
\uo{} & \rightarrow & p_{UO} = 0.5 \\
\pp{} & \rightarrow & p_{PP} = 0.35 \\
\sr{} & \rightarrow & p_{SR} = 0.15
\end{array}
\end{equation*} 
\item I tempi di servizio sono distribuiti esponenzialmente.
\item I clienti afferenti ad una stessa coda vengono serviti in accordo ad una disciplina FIFO (First-In, First-Out).
\item Il servizio di un cliente non può essere interrotto per favorire l'avanzamento di un altro con priorità superiore.
\end{itemize}

La politica di scheduling del sistema assume caratteristiche tipiche sia dello scheduling size-based che di quello astratto. In particolare:
\begin{itemize}
\item Ha caratteristiche astratte perché la probabilità di ricadere in una delle classi di priorità è funzione della frequenza con cui l'utente prende quella tipologia di ticket e non del tempo necessario all'elaborazione della richiesta da parte del servente. Infatti, il tempo necessario ad elaborare un ticket della classe $i$ non è necessariamente compreso in un intervallo $(x_{i-1}, x_i]$ bensì può assumere valori su tutto il semiasse reale positivo. 
\item Ha caratteristiche size-based perché classi di priorità diverse non condividono lo stesso tempo di servizio medio, ovvero $\exists\ i,j\ t.c.\ E[S_i] \neq E[S_j]$.
\end{itemize}

Al fine di agevolare la comprensione del funzionamento dello scheduler di sistema, di seguito sono riportati gli pseudocodici \ref{alg:modello-specifiche-1} e \ref{alg:modello-specifiche-2} che descrivono, rispettivamente, il comportamento del servente generico e di quello dedicato.

\begin{algorithm}
\scaption{Algoritmo di schedulazione del servente generico}
\label{alg:modello-specifiche-1}
\begin{algorithmic}[1]
\Procedure{GeneralPurposeServer}{}
\While{true}
	\If{customer owns \textsl{BancoPosta}}
		\If{\textsl{UnicaOperazioneBP} queue not empty}
			\State \textit{processes the first ticket of that type}
		\Else
			\If{\textsl{PagamentiPrelieviBP} queue not empty}
			\State \textit{processes the first ticket of that type}
			\EndIf
		\EndIf
	\Else
		\If{\textsl{UnicaOperazione} queue not empty}
			\State \textit{processes the first ticket of that type}
		\Else
			\If{\textsl{PagamentiPrelievi} queue not empty}
			\State \textit{processes the first ticket of that type}
			\EndIf
		\EndIf
	\EndIf
\EndWhile
\end{algorithmic}
\end{algorithm}

\begin{algorithm}
\scaption{Algoritmo di schedulazione del servente dedicato a ticket \sr{}}
\label{alg:modello-specifiche-2}
\begin{algorithmic}[1]
\Procedure{DedicatedServer}{}
\While{true}
	\If{customer owns \textsl{BancoPosta}}
		\If{\textsl{SpedizioneRitiriBP} queue not empty}
			\State \textit{processes the first ticket of that type}
		\Else
			\If{\textsl{UnicaOperazioneBP} queue not empty}
				\State \textit{processes the first ticket of that type}
			\Else
				\If{\textsl{PagamentiPrelieviBP} queue not empty}
					\State \textit{processes the first ticket of that type}
				\EndIf
			\EndIf
		\EndIf
	\Else
		\If{\textsl{SpedizioneRitiri} queue not empty}
			\State \textit{processes the first ticket of that type}
		\Else
			\If{\textsl{UnicaOperazione} queue not empty}
				\State \textit{processes the first ticket of that type}
			\Else
				\If{\textsl{PagamentiPrelievi} queue not empty}
					\State \textit{processes the first ticket of that type}
				\EndIf
			\EndIf
		\EndIf
	\EndIf
\EndWhile
\end{algorithmic}
\end{algorithm}

Assumendo che:
\begin{itemize}
\item Il tempo medio di servizio $E[S]$ sia pari a $10\ min$.
\item Il tempo medio di servizio per una richiesta di tipo \pp{} sia pari ad una volta e mezzo quello di una \uo{}, ovvero $E[S_{PP}] = 1.5 \cdot E[S_{UO}]$.
\item Il tempo medio di servizio per una richiesta di tipo \sr{} sia pari a due volte quello di una \uo{}, ovvero $E[S_{SR}] = 2 \cdot E[S_{UO}]$.
\end{itemize}
è possibile ricavare gli $E[S_i]$, con $i \in \lbrace UO, PP, SR \rbrace$, come segue:
\begin{equation}
\begin{split}
E[S] &= p_{UO}\cdot E[S_{UO}] + p_{PP}\cdot E[S_{PP}] + p_{SR}\cdot E[S_{SR}] \\
&= p_{UO}\cdot E[S_{UO}] + p_{PP}\cdot 1.5\cdot E[S_{UO}] + p_{SR}\cdot 2\cdot E[S_{UO}] \\
&= (p_{UO} + 1.5\cdot p_{PP} + 2\cdot p_{SR}) \cdot E[S_{UO}]
\end{split}
\end{equation}
da cui:
\begin{equation}
\begin{cases}
E[S_{UO}] = \frac{E[S]}{(p_{UO} + 1.5\cdot p_{PP} + 2\cdot p_{SR})} = \frac{400}{53} \simeq 7.5471698\ min \\[1em]
E[S_{PP}] = 1.5\cdot E[S_{UO}] = \frac{600}{53} \simeq 11.3207547\ min \\[1em]
E[S_{SR}] = 2 \cdot E[S_{UO}] = \frac{800}{53} \simeq 15.0943396\ min
\end{cases}
\end{equation}

I risultati ottenuti possono essere verificati con il seguente consistency check:
\begin{equation}
p_{UO}\cdot E[S_{UO}] + p_{PP}\cdot E[S_{PP}] + p_{SR}\cdot E[S_{SR}] = 10\ min = E[S]\qquad \text{\color{forestgreen}\textbf{OK} \checkmark}
\end{equation}

\begin{table}[ht]
\centering
{\tablecolors
\begin{tabular}{| l | r |}
\hline
Grandezza & Valore misurato nel 2019 \\
\hline
N$^o$ di clienti al giorno & 1.4 mln \\
\hline
N$^o$ di uffici postali & 12 809 \\
\hline
\end{tabular}}
\scaption{Estratto dei dati di interesse a partire dalla fonte citata}
\label{table:modello-specifiche-1}
\end{table}

Per ricavare il throughput del sistema $X$ si è fatto uso dei dati dati provenienti da \textsl{"Principali dati economici e finanziari di Poste Italiane"}\footnote{\url{https://www.posteitaliane.it/it/performance-finanziaria.html}} nel modo seguente:
\begin{equation}
X = \frac{\# \lbrace \text{clienti al giorno} \rbrace}{\# \lbrace \text{uffici postali} \rbrace} \cdot \frac{1}{24 \cdot 60} \simeq \frac{109.2981497}{24 \cdot 60} \simeq 0.0759015\ req/min
\end{equation} 




















\end{document}