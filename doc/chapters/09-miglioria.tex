\chapter{Miglioria dell'algoritmo di schedulazione}
Nelle sezioni che seguono vengono ripercorsi i passi dell'algoritmo seguito nei capitoli precedenti, al fine di presentare le differenze introdotte a seguito dell'adozione del nuovo meccanismo di schedulazione.
\section{Presentazione del caso di studio}
La descrizione del caso di studio rimane pressoché inalterata, ad eccezione della presenza di alcune regole addizionali per la scelta dei ticket da servire.

Definiti:
\begin{itemize}
\item $A$ un cliente titolare di un conto \textsl{BancoPosta}, possessore di un ticket di tipo \uo{} o \pp{}.
\item $B$ un cliente \textbf{non} titolare di un conto \textsl{BancoPosta}, possessore di un ticket di tipo \uo{} o \pp{}.
\item $C$ un cliente possessore di un ticket di tipo \sr{}.
\end{itemize}
Di seguito, sono riportate le regole aggiuntive:
\begin{itemize}
\item Dopo un certo numero di clienti di tipo $A$ serviti, gli sportelli non dedicati devono necessariamente servire, se presenti, altri di tipo $B$, privilegiando la fila più lunga. 
\item Dopo un certo numero di clienti di tipo $C$ serviti, lo sportello dedicato deve necessariamente servire, se presenti, altri di tipo $B$, privilegiando la fila più lunga.
\item Nel caso in cui non vi siano clienti di tipo $C$ da servire, lo sportello dedicato dà priorità a quelli di tipo $B$, privilegiando la fila più lunga.
\end{itemize}
\section{Obiettivi dello studio}
L'obiettivo di questo studio migliorativo è quello di diminuire, rispetto ai risultati precedentemente ottenuti, il numero degli sportelli da mantenere operativi in un'intera giornata lavorativa, continuando a garantire il rispetto dei requisiti di qualità stabiliti nel capitolo \ref{chp:obiettivi}.
\section{Modello concettuale}
Il modello concettuale proposto nel capitolo \ref{chp:modello-concettuale} continua ad essere rappresentativo anche a seguito dell'introduzione della miglioria. 

Pertanto:
\begin{itemize}
\item Le variabili che descrivono univocamente, ad ogni istante di tempo, lo stato del sistema
\item Le assunzioni alla base del modello concettuale
\end{itemize}
sono le medesime introdotte in tale capitolo.
\section{Modello delle specifiche}
Dal modello delle speficiche descritto nel capitolo \ref{chp:modello-specifiche}:
\begin{itemize}
\item Le variabili matematiche e le equazioni che le legano
\item Le assunzioni di base
\item I parametri di input del sistema
\end{itemize}
rimangono validi per il sistema, anche a seguito dell'introduzione della miglioria.

Di seguito, sono riportate le versioni rivisitate degli algoritmi di schedulazione \ref{alg:miglioria-modello-specifiche-1} e \ref{alg:miglioria-modello-specifiche-2}.

\begin{algorithm}[ht]
\SetAlgoLined
\While{true}{
\textcolor{blue}{\uIf{reached threshold}{
		\textit{pick the longest queue between \texttt{UNICA\_OP\_STD} and \texttt{PAGAM\_PREL\_STD}}\;
		\textit{processes the first ticket inside that queue}\;
	}}
	\uElseIf{\texttt{UNICA\_OP\_BP} queue not empty}{
		\textit{processes the first ticket of that type}\;
	}
	\uElseIf{\texttt{PAGAM\_PREL\_BP} queue not empty}{
		\textit{processes the first ticket of that type}\;
	}
	\uElseIf{\texttt{UNICA\_OP\_STD} queue not empty}{
		\textit{processes the first ticket of that type}\;
	}
	\uElseIf{\texttt{PAGAM\_PREL\_STD} queue not empty}{
		\textit{processes the first ticket of that type}\;
	}
}
\caption{Algoritmo di schedulazione del servente generico (con {\color{blue}patch})}
\label{alg:miglioria-modello-specifiche-1}
\end{algorithm}

\begin{algorithm}
\While{true}{
	\textcolor{blue}{\uIf{reached threshold and there are tickets \texttt{UNICA\_OP\_STD} or \texttt{PAGAM\_PREL\_STD}}{
		\textit{pick the longest queue between \texttt{UNICA\_OP\_STD} and \texttt{PAGAM\_PREL\_STD}}\;
		\textit{processes the first ticket inside that queue}\;
	}}
	\uElseIf{\texttt{SPED\_RIT\_BP} queue not empty}{
		\textit{processes the first ticket of that type}\;
	}
	\uElseIf{\texttt{SPED\_RIT\_STD} queue not empty}{
		\textit{processes the first ticket of that type}\;
	}
	\textcolor{blue}{\uElseIf{there are tickets \texttt{UNICA\_OP\_STD} or \texttt{PAGAM\_PREL\_STD}}{
		\textit{pick the longest queue between them}\;
		\textit{processes the first ticket inside that queue}\;
	}}
	\uElseIf{\texttt{UNICA\_OP\_BP} queue not empty}{
		\textit{processes the first ticket of that type}\;
	}
	\uElseIf{\texttt{PAGAM\_PREL\_BP} queue not empty}{
		\textit{processes the first ticket of that type}\;
	}
}
\caption{Algoritmo di schedulazione del servente dedicato (con {\color{blue}patch})}
\label{alg:miglioria-modello-specifiche-2}
\end{algorithm}
