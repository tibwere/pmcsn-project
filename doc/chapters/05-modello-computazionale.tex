\chapter{Modello computazionale}\label{chp:modello-computazionale}
L'approccio utilizzato per la realizzazione del modello computazionale è quello della next-event simulation. Di seguito è riportata un'analisi delle sue fasi.
\section{Stato del sistema}\label{sec:modello-computazionale-stato}
Le variabili di programma utilizzate per descrivere univocamente lo stato del sistema ad ogni istante di tempo sono:
\begin{itemize}
\item \texttt{{\color{code_purple} int} customers[NUMBER\_OF\_QUEUES]}: vettore di sei interi per la memorizzazione del numero totale di clienti per ciascuna tipologia di arrivo.
\item \texttt{{\color{code_purple} int} in\_service[NUMBER\_OF\_QUEUES]}: vettore di sei interi per la memorizzazione del numero di clienti in servizio per ciascuna tipologia di arrivo.
\item \texttt{{\color{code_purple} int} gen\_status[M-1]}: vettore di $M-1$ interi, che assumono valori \texttt{0} e \texttt{1}, per la memorizzazione dello stato dei serventi generali.
\item \texttt{{\color{code_purple} int} ded\_status}: intero, che assume valori \texttt{0} e \texttt{1}, per la memorizzazione dello stato del servente dedicato.
\end{itemize}

Il mapping fra le variabili matematiche definite nel modello delle specifiche e quelle di programma a livello computazionale è riassunto nella tabella \ref{table:modello-computazionale-1}. 

\begin{table}[ht]
\centering
{\tablecolors
\begin{tabular}{| l | l |}
\hline
Variabile matematica & Variabile di programma \\
\hline
$Customers_c(t)$ & \texttt{{\color{code_purple}int} customers[c]} \\
\hline
$InService_c(t)$ & \texttt{{\color{code_purple}int} in\_service[c]} \\
\hline
$Server_r(t)$ & \texttt{{\color{code_purple}int} gen\_status[r]} \\
\hline
$Server_\ded(t)$ & \texttt{{\color{code_purple}int} ded\_status} \\
\hline
\end{tabular}}
\caption{Mapping tra il modello delle specifiche e quello computazionale}
\label{table:modello-computazionale-1}
\end{table}
\newpage
A livello computazionale le classi di utenza, finora descritte in linguaggio naturale, sono state mappate negli indici numerici riportati in tabella \ref{table:modello-computazionale-2}.
\begin{table}[ht]
\centering
{\tablecolors
\begin{tabular}{| l | r |}
\hline
Classe di utenza & Indice numerico \\
\hline
\uo{} \textsl{BancoPosta} & \texttt{0} \\
\hline
\pp{} \textsl{BancoPosta} & \texttt{1} \\
\hline
\uo{} \textsl{Standard} & \texttt{2} \\
\hline
\pp{} \textsl{Standard} & \texttt{3} \\
\hline
\sr{} \textsl{BancoPosta} & \texttt{4} \\
\hline
\sr{} \textsl{Standard} & \texttt{5} \\
\hline
\end{tabular}}
\caption{Indici per gli array relativi alle classi di utenza}
\label{table:modello-computazionale-2}
\end{table}

Ad inizio simulazione lo stato del sistema viene impostato come segue:
\begin{center}
\begin{tabular}{l l l l l l}
\texttt{customers[c]} & \texttt{=} & \texttt{0;} & \texttt{in\_service[c]} & \texttt{=} & \texttt{0;} \\
\texttt{gen\_status[r]} & \texttt{=} & \texttt{0;} & \texttt{ded\_status} & \texttt{=} & \texttt{0;}
\end{tabular}
\end{center}
per ogni $\mathtt{c\in \lbrace 0, 1, \dots, 5 \rbrace}$ e per ogni $\mathtt{r\in \lbrace 0, 1, \dots, M-2 \rbrace}$.

\section{Eventi}\label{sec:modello-computazionale-eventi}
Lo stato del sistema può cambiare all'occorrenza delle seguenti tipologie di eventi propri:
\begin{itemize}
\item Arrivo di un cliente afferente alla classe $c$, che comporta:
\begin{itemize}
\item \texttt{customers[c] = customers[c] + 1}
\end{itemize}
\newpage
\item Completamento del servizio di un cliente della classe $c$ da parte di un server generale $r$, che comporta:
\begin{itemize}
\item Il passaggio di \texttt{gen\_status[r]} dal valore \texttt{1} a \texttt{0}
\item \texttt{customers[c] = customers[c] - 1}
\item \texttt{in\_service[c] = in\_service[c] - 1}
\end{itemize}
\item Completamento del servizio di un cliente della classe $c$ da parte del server dedicato, che comporta:
\begin{itemize}
\item Il passaggio di \texttt{ded\_status} dal valore \texttt{1} a \texttt{0}
\item \texttt{customers[c] = customers[c] - 1}
\item \texttt{in\_service[c] = in\_service[c] - 1}
\end{itemize}
\end{itemize} 
\section{Clock di simulazione}\label{sec:modello-computazionale-clock}
L'orologio di sistema adottato nel simulatore è contenuto all'interno di una struttura dati\footnote{Nel codice viene adottata la variabile globale \texttt{{\color{code_purple}times\_t} t}.}, riportata nello snippet \ref{lst:modello-computazionale-1}, che memorizza informazioni relative al tempo di simulazione. In particolare:
\begin{itemize}
\item \texttt{{\color{code_purple}double} next}: istante di occorenza del successivo evento nel sistema.
\item \texttt{{\color{code_purple}double} last[NUMBER\_OF\_QUEUES]}: ciascuna entry \texttt{c} rappresenta l'istante dell'ultimo arrivo relativo alla classe di utenza $c$.
\item \texttt{{\color{code_purple}double} current}: orologio di sistema.
\end{itemize}

\begin{lstlisting}[label={lst:modello-computazionale-1}, caption={Struttura dati per la gestione del tempo}]
typedef struct times {
    double next;                 	                                         
    double current;                	
    double last[NUMBER_OF_QUEUES];  
} times_t;
\end{lstlisting}

Di seguito sono riportate alcune osservazioni relative al tempo di simulazione:
\begin{itemize}
\item L'unità di misura di riferimento è il minuto.
\item L'orologio di sistema (\texttt{t->current}) viene inizializzato a \texttt{START}.
\item Il tempo di simulazione $\tau$ è pari a \texttt{STOP}.
\item Nel caso in cui l'istante $t^*$ d'occorrenza di un arrivo sia postumo al termine della simulazione ($t^* > \tau$), il successivo della medesima tipologia viene impostato pari a \texttt{INFTY} per denotare il blocco di tale flusso di arrivi (\textit{"close the door"}).
\end{itemize}
Per il caso di studio in analisi è stata adottata la seguente configurazione di parametri:
\begin{center}
\begin{tabular}{l c r}
\texttt{START = 0}, & \texttt{STOP = 480}, & \texttt{INFTY = 100 $\cdot$ STOP}
\end{tabular}
\end{center}
dove il valore di \texttt{STOP} è pari al periodo di erogazione dei ticket espresso in minuti, in accordo a quanto specificato nella presentazione del caso di studio (cap. \ref{chp:presentazione}).
\section{Scheduler}\label{sec:modello-computazionale-scheduler}
Il meccanismo di avanzamento del tempo, che garantisce l'occorrenza ordinata degli eventi temporali, ad ogni iterazione del loop di simulazione:
\begin{enumerate}
\item Computa l'occorrenza del next-event (\texttt{t->next = next\_event(\dots)})
\item Aggiorna l'orologio di sistema (\texttt{t->current = t->next})
\end{enumerate}
fintantoché è verificata almeno una delle seguenti due condizioni:
\begin{itemize}
\item Il tempo di occorrenza degli eventi è non superiore a $\tau$.
\item Sono presenti clienti nel sistema non ancora serviti.
\end{itemize}

\section{Lista degli eventi}\label{sec:modello-computazionale-lista-eventi}
La lista degli eventi, a livello implementativo, è gestita mediante le strutture dati\footnote{Nel codice viene adottata la variabile globale \texttt{{\color{code_purple}event\_list\_t} events}.} riportata nello snippet \ref{lst:modello-computazionale-2}.
Ciascun campo della struttura mantiene gli istanti di tempo d'occorrenza per una categoria di eventi. In particolare:
\begin{itemize}
\item \texttt{{\color{code_purple}double} arrivals[NUMBER\_OF\_QUEUES]}: ciascuna entry \texttt{c} rappresenta l'istante del successivo arrivo relativo alla classe d'utenza $c$.
\item \texttt{{\color{code_purple}compl\_event\_t} gen\_completions[M-1]}: ciascuna entry \texttt{r} mantiene l'istante del prossimo completamento del server \texttt{r} e la classe dell'utente in servizio su esso.
\item \texttt{{\color{code_purple}compl\_event\_t} ded\_completion}: mantiene l'istante del prossimo completamento del server dedicato e la classe dell'utente in servizio su esso.
\end{itemize}
\newpage
\begin{lstlisting}[label={lst:modello-computazionale-2}, caption={Struttura dati per la lista degli eventi}]
typedef struct compl_event {
    double time;
    int user_class;
} compl_event_t;

typedef struct event_list {
    double arrivals[NUMBER_OF_QUEUES];  
    compl_event_t *gen_completions[M-1];      
    compl_event_t *ded_completion;      
} event_list_t; 
\end{lstlisting}

\section{Generazione dei nuovi eventi}\label{sec:modello-computazionale-gen-nuovi-eventi}
Di seguito è riportata un'analisi della metodologia adottata nella generazione degli istanti di tempo pseudocasuali. A tal proposito, è stata utilizzata la libreria \href{http://www.math.wm.edu/~leemis/}{qui} reperibile, distribuita durante il corso.

In particolare:
\begin{itemize}
\item Una sola volta, all'avvio del simulatore, viene invocata la funzione \texttt{PlantSeeds()} per la generazione dei flussi. Dei 256 generati ne vengono utilizzati 9:
\begin{itemize}
\item 6 per generare gli istanti d'arrivo per ciascuna classe d'utenza.
\item 3 per generare gli istanti di completamento per ciascuna tipologia di ticket.
\end{itemize}

\item Nella funzione \texttt{GetArrival} in base alla classe d'utenza passata come parametro:
\begin{itemize}
\item Viene selezionato lo stream mediante la funzione \texttt{SelectStream}.
\item Viene generata una variata pseudocasuale di tipo esponenziale, la cui media è funzione della classe selezionata:
\begin{center}
\begin{tabular}{l l}
\textsl{Unica Op. BancoPosta} & $\to$ \texttt{Exponential(1.0/(LAMBDA*P\_UO*P\_BP))} \\
\textsl{Pagam. \& Prel. BancoPosta} & $\to$ \texttt{Exponential(1.0/(LAMBDA*P\_PP*P\_BP))} \\
\textsl{Unica Op. Standard} & $\to$ \texttt{Exponential(1.0/(LAMBDA*P\_UO*(1-P\_BP)))} \\
\textsl{Pagam. \& Prel. Standard} & $\to$ \texttt{Exponential(1.0/(LAMBDA*P\_PP*(1-P\_BP)))} \\
\textsl{Sped. \& Rit. BancoPosta} & $\to$ \texttt{Exponential(1.0/(LAMBDA*P\_SR*P\_BP))} \\
\textsl{Sped. \& Rit. BancoPosta} & $\to$ \texttt{Exponential(1.0/(LAMBDA*P\_SR*(1-P\_BP)))} \\
\end{tabular}
\end{center}
\item Viene aggiornato l'istante del successivo arrivo di un cliente afferente a tale classe.
\end{itemize}

\item Nella funzione \texttt{GetService} in base alla classe d'utenza passata come parametro:
\begin{itemize}
\item Viene selezionato lo stream mediante la funzione \texttt{SelectStream}.
\item Viene generata una variata pseudocasuale di tipo esponenziale, la cui media è funzione della classe selezionata:
\begin{equation*}
\begin{array}{l l l}
\begin{rcases*}
\textsl{Unica Op. BancoPosta} \\
\textsl{Unica Op. Standard}
\end{rcases*}
& \to & \mathtt{Exponential(1.0/MU\_UO)} \\[2em]
\begin{rcases*}
\textsl{Pagam. \& Prel. BancoPosta} \\
\textsl{Pagam. \& Prel. Standard}
\end{rcases*}
& \to & \mathtt{Exponential(1.0/MU\_PP)} \\[2em]
\begin{rcases*}
\textsl{Sped. \& Rit. BancoPosta} \\
\textsl{Sped. \& Rit. Standard}
\end{rcases*}
& \to & \mathtt{Exponential(1.0/MU\_SR)}
\end{array}
\end{equation*}
\item Viene aggiornato l'istante del successivo completamento di un cliente afferente a tale classe.
\end{itemize}
\end{itemize}
È opportuno osservare che:
\begin{itemize}
\item Nelle funzioni \texttt{GetArrival} e \texttt{GetService} vengono generate variate esponenziali, in accordo a quanto assunto nel modello delle specifiche (cap. \ref{chp:modello-specifiche}).
\item Per ciascuna tipologia di ticket, la titolarità di un conto \textsl{BancoPosta} non ha influenza sul tempo di servizio.
\item Le macro:
\begin{itemize}
\item \texttt{MU\_UO}, \texttt{MU\_PP} e \texttt{MU\_SR}
\item \texttt{P\_UO}, \texttt{P\_PP}, \texttt{P\_SR} e \texttt{P\_BP}
\item \texttt{LAMBDA}
\end{itemize}
vengono utilizzate nel codice facendo riferimento ai valori fissati nel modello delle specifiche (cap. \ref{chp:modello-specifiche}). 
\end{itemize}

\section{Algoritmo next-event}\label{sec:modello-computazionale-algoritmo}
Di seguito sono analizzati alcuni statement di codice per l'implementazione dell'algoritmo di simulazione:
\begin{enumerate}[label=Step \arabic*), align=left, leftmargin=*]
\item \textbf{Inizializzazione}
\begin{itemize}
\item L'orologio di simulazione è inizializzato come descritto nella sezione \ref{sec:modello-computazionale-clock}.
\item La lista degli eventi (\texttt{events}) è inizializzata determinando la prima occorrenza di ogni possibile evento. In particolare:
\begin{itemize}
\item Per ogni classe d'utenza $c$ viene generato il primo istante di arrivo, come descritto nella sezione \ref{sec:modello-computazionale-gen-nuovi-eventi}.

\item Per ogni servente, sia esso generale o dedicato, viene settato a \texttt{INFTY} l'istante del prossimo completamento per denotare il fatto che esso non stia attualmente processando nessuna richiesta.
\end{itemize}
\end{itemize}
\item \textbf{Processamento evento corrente}
\begin{enumerate}
\item La lista degli eventi è scandita per determinare l'evento più imminente possibile, in accordo a quanto descritto nella sezione \ref{sec:modello-computazionale-scheduler}.
\item L'orologio di sistema viene avanzato al tempo di occorrenza dell'evento schedulato, in relazione a quanto proposto in sezione \ref{sec:modello-computazionale-scheduler}.
\item L'evento schedulato viene processato, come illustrato in sezione \ref{sec:modello-computazionale-eventi}.
\end{enumerate}
\item \textbf{Schedulazione nuovi eventi}
\begin{itemize}
\item Nel caso in cui l'evento corrente sia un arrivo di un cliente afferente ad una classe $c$ ed esiste un servente libero in grado di servirlo:
\begin{enumerate}
\item Lo stato del servente viene posto pari ad \texttt{1} (\texttt{BUSY}).
\item Il numero di clienti in servizio afferenti alla classe $c$ viene incrementato di una unità.
\item Viene generato l'istante di completamento di tale servizio, come descritto nella sezione \ref{sec:modello-computazionale-gen-nuovi-eventi}.
\end{enumerate}

\item Nel caso in cui l'evento corrente sia un completamento di un cliente afferente ad una classe $c$, il servente che si è appena liberato processa, se presente, il primo cliente in coda seguendo l'ordine di priorità, replicando l'iter proposto nel bullet precedente.
\end{itemize}

\item \textbf{Terminazione}
\begin{itemize}
\item L'evento artificiale che causa la terminazione della simulazione è l'intersezione dei seguenti:
\begin{itemize}
\item La prossima occorrenza di una qualsiasi tipologia di arrivo è postuma a $\tau$.
\item Non sono presenti richieste da processare.
\end{itemize}
Questo perché si è interessati allo studio delle prestazioni di un ufficio postale nell'arco della giornata lavorativa (capp. \ref{chp:presentazione} e \ref{chp:obiettivi}).

\end{itemize}
\end{enumerate}

\section{Campionamento delle statistiche}\label{sec:modello-computazionale-campionamento-stat}
Per ogni classe d'utenza $c$, le statistiche calcolate durante la simulazione sono:
\begin{itemize}
\item \textbf{Time-averaged}
\begin{itemize}
\item La popolazione media nel centro
\begin{equation}
\bar{l}_c=\frac{1}{t_{end}}\int_0^{t_{end}} l_c(t) dt
\end{equation}
\item La popolazione media nella coda
\begin{equation}
\bar{q}_c=\frac{1}{t_{end}}\int_0^{t_{end}} q_c(t) dt
\end{equation}
\item La popolazione media in servizio 
\begin{equation}
\bar{y}_c=\frac{1}{t_{end}}\int_0^{t_{end}} y_c(t) dt
\end{equation}
\end{itemize}
\item \textbf{Job-averaged}
\begin{itemize}
\item Il tempo medio di interarrivo
\begin{equation}
\bar{r}_c = \frac{1}{n_c} \sum_{i=1}^{n_c} r_{c,i} = \frac{a_{n_c}}{n}
\end{equation}
\item Il tempo medio di risposta
\begin{equation}
\bar{w}_c = \frac{1}{n_c} \sum_{i=1}^{n_c} w_{c,i}
\end{equation}
\item Il tempo medio d'attesa
\begin{equation}
\bar{d}_c = \frac{1}{n_c} \sum_{i=1}^{n_c} d_{c,i}
\end{equation}
\item Il tempo medio di servizio
\begin{equation}
\bar{s}_c = \frac{1}{n_c} \sum_{i=1}^{n_c} s_{c,i}
\end{equation}
\end{itemize}
\end{itemize}
dove $t_{end} \geq \tau$ è l'istante di fine simulazione.

Poiché le funzioni sono \textit{stepwise} gli integrali possono essere calcolati incrementando, ad ogni iterazione del ciclo di simulazione, le aree come segue:
\begin{lstlisting}[caption={Calcolo degli integrali}]
area->customers[c]+=(t->next-t->current)*customers[c];
area->queue[c]+=(t->next-t->current)*(customers[c]-in_service[c]);
area->service[c]+=(t->next-t->current)*in_service[c];
\end{lstlisting}

Inoltre:
\begin{itemize}
\item Nel caso di simulazione ad orizzonte finito, al termine di ogni replica 
\item Nel caso di simulazione ad orizzonte infinito, al termine di ogni batch
\end{itemize}
viene calcolata una delle statistiche sopra presentate e il risultato viene dato in input a \texttt{uvs} per il calcolo di media, deviazione standard, massimo e minimo oppure a \texttt{estimate} per il calcolo di un intervallo di confidenza al $95\%$. In entrambi i casi, i programmi utilizzano l'algoritmo di Welford.