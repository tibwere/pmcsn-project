\chapter{Esperimenti di simulazione}\label{chp:esperimenti-simulazione}
In accordo agli obiettivi dello studio, per la progettazione degli esperimenti di simulazione, è stata considerata unicamente l'analisi dello stato transiente del sistema. Infatti, perderebbe di significato considerare lo stato stazionario per determinare il numero minimo di serventi necessari affinché vengano soddisfatti i QoS descritti nel capitolo \ref{chp:obiettivi}. Questo perché in una giornata lavorativa, assunta pari ad otto ore, il sistema non riesce a raggiungere lo stato stazionario prima che si verifichi la condizione di scarico (\textit{"close the door"}) ed inoltre rimane forte l'influenza delle condizioni iniziali.