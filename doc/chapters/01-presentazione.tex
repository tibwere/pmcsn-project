\chapter{Presentazione del caso di studio}\label{chp:presentazione}
Il sistema oggetto dell'analisi in questione eroga le seguenti tipologie di servizi:
\begin{enumerate}
\item \uo{} (e.g. ricarica \textsl{PostePay}, invio raccomandata e pagamento di massimo tre bollettini)
\item \pp{} (e.g. pagamento di un numero arbitrario di bollettini, bollo auto e libretti)  
\item \sr{} (e.g. invio corrispondenza, lettere, pacchi e raccomandate)
\end{enumerate}

Per essere serviti i clienti possono recarsi all'ufficio postale, prendere un ticket relativo al servizio a cui sono interessati e mettersi in coda in attesa del proprio turno. Nel caso in cui essi dimostrino di essere titolari di un conto \textsl{BancoPosta} potranno accodarsi in una fila dedicata.

Un insieme di sportelli serve le richieste degli utenti in accordo alle seguenti regole: 
\begin{enumerate}[label=R\arabic*), align=left, leftmargin=*]
\item I ticket di tipo \sr{} vengono serviti da uno sportello dedicato il quale, in assenza di questa tipologia, opera come gli altri. Il comportamento di tale servente è schematizzato in figura \ref{fig:presentazione-1}. 
\item Poiché, per definizione, ticket di tipo \uo{} dovrebbero richiedere meno tempo per essere processati, viene assegnata loro una priorità maggiore di \pp{}.
\item I clienti titolari di un conto \textsl{BancoPosta} vengono serviti con una priorità maggiore rispetto agli altri, in accordo alle regole R1 ed R2.
\end{enumerate}

\begin{figure}[ht]
\centering
\includegraphics[width=0.75\linewidth]{presentazione-1}
\caption{Schema del comportamento del servente dedicato ai ticket di tipo \sr}
\label{fig:presentazione-1}
\end{figure}
