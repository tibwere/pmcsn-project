\chapter{Obiettivi dello studio}\label{chp:obiettivi}
L'obiettivo dello studio è quello di minimizzare il numero degli sportelli attivi, al fine di garantire il soddisfacimento di differenti requisiti di qualità per ciascuna tipologia di servizio illustrata nella presentazione del caso di studio (cap. \ref{chp:presentazione}):

\begin{enumerate}[label=QoS-\arabic*), align=left, leftmargin=*]
\item I titolari di un conto \textsl{BancoPosta} devono sperimentare un tempo d'attesa non superiore a due volte il tempo medio di servizio per la tipologia di ticket in loro possesso.
\item I clienti non titolari di un conto \textsl{BancoPosta}:
\begin{enumerate}
\item Possessori di ticket di tipo \uo{} devono sperimentare un tempo d'attesa non superiore a quattro volte il tempo medio di servizio dei clienti della medesima tipologia.
\item Possessori di ticket di tipo \pp{} devono sperimentare un tempo d'attesa non superiore a cinque volte il tempo medio di servizio dei clienti della medesima tipologia.
\item Possessori di ticket di tipo \sr{} devono sperimentare un tempo d'attesa non superiore a tre volte il tempo medio di servizio dei clienti della medesima tipologia.
\end{enumerate}
\end{enumerate}
