\chapter{Modello delle Specifiche}\label{chp:modello-specifiche}
Le seguenti variabili matematiche, definite per ogni istante di tempo $t$, identificano univocamente una rappresentazione a livello delle specifiche dello stato del sistema:

\begin{itemize}
\item Lo stato dello sportello $r$-esimo è dato da:
\begin{equation}
\label{eqn:modello-specifiche-3}
Server_r(t) \in \lbrace 0, 1 \rbrace
\end{equation}
\item Per ciascuna classe $c$ il numero totale di clienti è definito come:
\begin{equation}
\label{eqn:modello-specifiche-2}
Customers_c(t) \in \mathbb{N}
\end{equation}
\item Per ciascuna classe $c$ il numero di clienti in servizio è definito come segue:
\begin{itemize}
\item Per $c$ diversa da \sr{} \textsl{BancoPosta} e \sr{} \textsl{Standard}:
\begin{equation}
\label{eqn:modello-specifiche-2}
InService_c(t) \in \lbrace 0, 1, \dots, M \rbrace
\end{equation}
\item Altrimenti:
\begin{equation}
\label{eqn:modello-specifiche-2}
InService_c(t) \in \lbrace 0, 1 \rbrace
\end{equation}
\end{itemize}
\end{itemize}

Dalle variabili appena descritte, è immediato ricavare il numero di clienti in coda per ciascuna classe di utenza $c$:
\begin{equation}
Queue_c(t) = Customers_c(t) - InService_c(t)
\end{equation}

Di seguito sono riportate alcune assunzioni che saranno alla base di questa e delle successive fasi dello studio:
\begin{itemize}
\item I clienti arrivano all'ufficio postale ad istanti di tempo casuali, il che implica:
\begin{itemize}
\item Distribuzione poissoniana degli arrivi.
\item Distribuzione esponenziale dei tempi di interarrivo.
\end{itemize}
\item La probabilità che un cliente sia titolare di un conto \textsl{BancoPosta} è pari a $p_{BP} = 0.25$.
\item Le probabilità con cui ciascuna tipologia di ticket viene acquisita sono le seguenti:
\begin{equation*}
\begin{array}{l c l}
\uo{} & \rightarrow & p_{UO} = 0.5 \\
\pp{} & \rightarrow & p_{PP} = 0.35 \\
\sr{} & \rightarrow & p_{SR} = 0.15
\end{array}
\end{equation*} 
\item I tempi di servizio sono distribuiti esponenzialmente.
\item I clienti afferenti ad una stessa coda vengono serviti in accordo ad una disciplina FIFO (First-In, First-Out).
\item Il servizio di un cliente non può essere interrotto per favorire l'avanzamento di un altro con priorità superiore.
\end{itemize}

Al fine di agevolare la comprensione del funzionamento dello scheduler di sistema, di seguito sono riportati gli pseudocodici \ref{alg:modello-specifiche-1} e \ref{alg:modello-specifiche-2} che descrivono, rispettivamente, il comportamento del servente generico e di quello dedicato.

\begin{algorithm}[ht]
\SetAlgoLined
\While{true}{
	\uIf{\texttt{UNICA\_OP\_BP} queue not empty}{
		\textit{processes the first ticket of that type}\;
	}
	\uElseIf{\texttt{PAGAM\_PREL\_BP} queue not empty}{
		\textit{processes the first ticket of that type}\;
	}
	\uElseIf{\texttt{UNICA\_OP\_STD} queue not empty}{
		\textit{processes the first ticket of that type}\;
	}
	\uElseIf{\texttt{PAGAM\_PREL\_STD} queue not empty}{
		\textit{processes the first ticket of that type}\;
	}
	\uElse{
		\textit{do nothing}\;
	}
}
\caption{Algoritmo di schedulazione del servente generico}
\label{alg:modello-specifiche-1}
\end{algorithm}

\begin{algorithm}
\SetAlgoLined
\While{true}{
	\uIf{\texttt{SPED\_RIT\_BP} queue not empty}{
		\textit{processes the first ticket of that type}\;
	}
	\uElseIf{\texttt{SPED\_RIT\_STD} queue not empty}{
		\textit{processes the first ticket of that type}\;
	}
	\uElseIf{\texttt{UNICA\_OP\_BP} queue not empty}{
		\textit{processes the first ticket of that type}\;
	}
	\uElseIf{\texttt{PAGAM\_PREL\_BP} queue not empty}{
		\textit{processes the first ticket of that type}\;
	}
	\uElseIf{\texttt{UNICA\_OP\_STD} queue not empty}{
		\textit{processes the first ticket of that type}\;
	}
	\uElseIf{\texttt{PAGAM\_PREL\_STD} queue not empty}{
		\textit{processes the first ticket of that type}\;
	}
	\uElse{
		\textit{do nothing}\;	
	}
}
\caption{Algoritmo di schedulazione del servente dedicato}
\label{alg:modello-specifiche-2}
\end{algorithm}

\newpage
In tabella \ref{table:modello-specifiche-1} sono riportati tempi di servizio medi assunti per ciascuna tipologia di ticket, sperimentati su un singolo servente.
\begin{table}[ht]
\centering
{\tablecolors
\begin{tabular}{| l | r |}
\hline
Tipologia di ticket & Tempo di servizio medio \\
\hline
\uo{} & 7 min \\
\hline
\pp{} & 14 min \\
\hline
\sr{} & 10 min \\
\hline
\end{tabular}}
\caption{Assunzioni sui tempi medi di servizio sul singolo sportello}
\label{table:modello-specifiche-1}
\end{table}	

\begin{table}[ht]
\centering
{\tablecolors
\begin{tabular}{| l | r |}
\hline
Grandezza & Valore misurato nel 2018 \\
\hline
N$^o$ di clienti al giorno & 1.5 mln \\
\hline
N$^o$ di uffici postali & 12 812 \\
\hline
\end{tabular}}
\caption{Estratto dei dati di interesse a partire dalla fonte citata}
\label{table:modello-specifiche-2}
\end{table}

Per ricavare il throughput dell'intero sistema si è fatto uso dei dati dati provenienti da \textsl{"Principali dati economici e finanziari di Poste Italiane"}\footnote{\url{https://www.posteitaliane.it/it/performance-finanziaria.html}} nel modo seguente:
\begin{equation}
X = \frac{\# \text{clienti al giorno}}{\#\text{uffici postali}} = \frac{1500000}{12812}\ req/wd = \frac{1500000}{12812\cdot 480} \simeq 0.243912\ req/min
\end{equation} 
dove la conversione da giornata lavorativa (\textsl{wd}) a minuti effettivi di lavoro è stata realizzata in accordo a quanto specificato negli obiettivi (cap. \ref{chp:obiettivi}).

È opportuno osservare che, sotto l'ipotesi di \textsl{job flow balance}, la frequenza d'arrivo media al centro $\lambda$ coincide con il throughput $X$.

