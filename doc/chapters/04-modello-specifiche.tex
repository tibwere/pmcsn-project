\chapter{Modello delle Specifiche}\label{chp:modello-specifiche}
Ciascuna tipologia di ticket, definita tramite la coppia di pedici $(h,j)$ appartenente al prodotto cartesiano 
\begin{equation}
R = \lbrace \mathtt{BANCO\_POSTA},\ \mathtt{STANDARD}\rbrace \times \lbrace \mathtt{UNICA\_OP},\mathtt{PAGAM\_PREL},\ \mathtt{SPED\_RIT} \rbrace
\end{equation}
viene mappata su valori numerici, in accordo alla tabella \ref{table:modello-specifiche-1a}.

\begin{table}[ht]
\centering
\begin{subtable}{0.5\textwidth}
\centering
{\tablecolors
\begin{tabular}{| l | l | r |}
\hline
$h$ & $j$ & Valore\\
\hline
$\mathtt{BANCO\_POSTA}$ & $\mathtt{UNICA\_OP}$ & $0$\\
\hline
$\mathtt{BANCO\_POSTA}$ & $\mathtt{PAGAM\_PREL}$ & $1$\\
\hline
$\mathtt{STANDARD}$ & $\mathtt{UNICA\_OP}$ & $2$\\
\hline
$\mathtt{STANDARD}$ & $\mathtt{PAGAM\_PREL}$ & $3$\\
\hline
$\mathtt{BANCO\_POSTA}$ & $\mathtt{SPED\_RIT}$ & $4$\\
\hline
$\mathtt{STANDARD}$ & $\mathtt{SPED\_RIT}$ & $5$\\
\hline
\end{tabular}}
\caption{Tipologia di ticket}
\label{table:modello-specifiche-1a}
\end{subtable}%
\begin{subtable}{0.5\textwidth}
\centering
{\tablecolors
\begin{tabular}{| l | r |}
\hline
Stato del server & Valore\\
\hline
$\mathtt{IDLE}$ & $-1$\\
\hline
$\mathtt{UNICA\_OP\_BP}$ & $0$\\
\hline
$\mathtt{PAGAM\_PREL\_BP}$ & $1$\\
\hline
$\mathtt{UNICA\_OP\_STD}$ & $2$\\
\hline
$\mathtt{PAGAM\_PREL\_STD}$ & $3$\\
\hline
$\mathtt{SPED\_RIT\_BP}$ & $4$\\
\hline
$\mathtt{SPED\_RIT\_STD}$ & $5$\\
\hline
\end{tabular}}
\caption{Stato del server}
\label{table:modello-specifiche-1b}
\end{subtable}
\caption{Mapping tra il modello concettuale e quello delle specifiche}
\label{table:modello-specifiche-1}
\end{table}

In particolare, è facile osservare come le righe delle due tabelle nella \ref{table:modello-specifiche-1} siano tra loro relazionate, ovvero lo stato del server coincide sempre con la tipologia di ticket in servizio, ad eccezione del caso in cui esso sia \texttt{IDLE}. Per comodità, tale relazione è riportata nella tabella \ref{table:modello-specifiche-2}.

\begin{table}[ht]
\centering
{\tablecolors
\begin{tabular}{| l || l | l || r |}
\hline
Stato del server & $h$ & $j$ & Valore\\
\hline
$\mathtt{IDLE}$ & - & - & $-1$\\
\hline
$\mathtt{UNICA\_OP\_BP}$ & $\mathtt{BANCO\_POSTA}$ & $\mathtt{UNICA\_OP}$ & $0$\\
\hline
$\mathtt{PAGAM\_PREL\_BP}$ & $\mathtt{BANCO\_POSTA}$ & $\mathtt{PAGAM\_PREL}$ & $1$\\
\hline
$\mathtt{UNICA\_OP\_STD}$ & $\mathtt{STANDARD}$ & $\mathtt{UNICA\_OP}$ & $2$\\
\hline
$\mathtt{PAGAM\_PREL\_STD}$ & $\mathtt{STANDARD}$ & $\mathtt{PAGAM\_PREL}$ & $3$\\
\hline
$\mathtt{SPED\_RIT\_BP}$ & $\mathtt{BANCO\_POSTA}$ & $\mathtt{SPED\_RIT}$ & $4$\\
\hline
$\mathtt{SPED\_RIT\_STD}$ & $\mathtt{STANDARD}$ & $\mathtt{SPED\_RIT}$ & $5$\\
\hline
\end{tabular}}
\caption{\texttt{RIGHT JOIN} sulla colonna `Valore' delle tabelle \ref{table:modello-specifiche-1a} e \ref{table:modello-specifiche-1b}}
\label{table:modello-specifiche-2}
\end{table}

\newpage
Le seguenti variabili, definite:
\begin{itemize}
\item Per ogni istante di tempo $t$
\item Per ogni indice $i$ appartenente alla colonna \textsl{`Valore'} della tabella \ref{table:modello-specifiche-2}
\end{itemize}
identificano univocamente una rappresentazione a livello delle specifiche dello stato del sistema:

\begin{itemize}
\item Per ogni $i \neq -1$ il numero di clienti presenti nel sistema è dato da:
\begin{equation}
\label{eqn:modello-specifiche-2}
Customers_i(t) \in \mathbb{N}
\end{equation}
\item Lo stato dello sportello $r$-esimo (con $r \in \lbrace 1, 2, \dots, \ded-1, \ded+1, \dots, M \rbrace$) è dato da:
\begin{equation}
\label{eqn:modello-specifiche-3}
Server_r(t) \in \lbrace -1, 0, \dots, 3 \rbrace
\end{equation}
in accordo alla tabella \ref{table:modello-specifiche-2} (e.g. se il servente $r$-esimo sta elaborando un ticket di tipo \uo{} \textsl{BancoPosta} all'istante $t$, allora $Server_r(t)=0$).
\item Lo stato dello sportello dedicato è dato da:
\begin{equation}
\label{eqn:modello-specifiche-4}
Server_\ded(t) \in \lbrace -1, 0, \dots, 5 \rbrace
\end{equation}
in accordo alla tabella \ref{table:modello-specifiche-2}\footnote{La variabile del server $\ded$-esimo può assumere un insieme più vasto di valori perché è l'unico che può elaborare ticket di tipo \sr{}.}.
\end{itemize}

Dalle variabili appena descritte, è immediato ricavare il numero di richieste in coda per ciascuna tipologia di ticket:
\begin{equation}
Queue_i(t) = Customers_i(t) - \# \lbrace Server_v(t)\ \vert\ Server_v(t) = i \rbrace
\end{equation}
con $v \in \lbrace 1, 2, \dots, M \rbrace$, ovvero compreso anche il server dedicato.

Infine, per ricavare il numero di clienti totale nel sistema è possibile sfruttare la seguente relazione:
\begin{equation}
Customers(t) = \sum_{i=0}^{5} Customers_i(t)
\end{equation}

Di seguito sono riportate alcune assunzioni che saranno alla base di questa e delle successive fasi dello studio:
\begin{itemize}
\item I clienti arrivano all'ufficio postale ad istanti di tempo casuali, il che implica:
\begin{itemize}
\item Distribuzione poissoniana degli arrivi.
\item Distribuzione esponenziale dei tempi di interarrivo.
\end{itemize}
\item La probabilità che un cliente sia titolare di un conto \textsl{BancoPosta} è pari a $p_{BP} = 0.25$.
\item Le probabilità con cui ciascuna tipologia di ticket viene acquisita sono le seguenti:
\begin{equation*}
\begin{array}{l c l}
\uo{} & \rightarrow & p_{UO} = 0.5 \\
\pp{} & \rightarrow & p_{PP} = 0.35 \\
\sr{} & \rightarrow & p_{SR} = 0.15
\end{array}
\end{equation*} 
\item I tempi di servizio sono distribuiti esponenzialmente.
\item I clienti afferenti ad una stessa coda vengono serviti in accordo ad una disciplina FIFO (First-In, First-Out).
\item Il servizio di un cliente non può essere interrotto per favorire l'avanzamento di un altro con priorità superiore.
\end{itemize}

La politica di schedulazione del sistema assume caratteristiche tipiche sia dello scheduling size-based che di quello astratto. In particolare:
\begin{itemize}
\item Ha caratteristiche astratte perché la probabilità di ricadere in una delle classi di priorità è funzione della frequenza con cui l'utente prende quella tipologia di ticket e non del tempo necessario all'elaborazione della richiesta da parte del servente. Infatti, il tempo necessario ad elaborare un ticket della classe $i$ non è necessariamente compreso in un intervallo $(x_{i-1}, x_i]$ bensì può assumere valori su tutto il semiasse reale positivo. 
\item Ha caratteristiche size-based perché classi di priorità diverse non condividono lo stesso tempo di servizio medio, ovvero $\exists\ i,j\ t.c.\ E[S_i] \neq E[S_j]$.
\end{itemize}

Al fine di agevolare la comprensione del funzionamento dello scheduler di sistema, di seguito sono riportati gli pseudocodici \ref{alg:modello-specifiche-1} e \ref{alg:modello-specifiche-2} che descrivono, rispettivamente, il comportamento del servente generico e di quello dedicato.

\begin{algorithm}[ht]
\SetAlgoLined
\While{true}{
	\uIf{\texttt{UNICA\_OP\_BP} queue not empty}{
		\textit{processes the first ticket of that type}\;
	}
	\uElseIf{\texttt{PAGAM\_PREL\_BP} queue not empty}{
		\textit{processes the first ticket of that type}\;
	}
	\uElseIf{\texttt{UNICA\_OP\_STD} queue not empty}{
		\textit{processes the first ticket of that type}\;
	}
	\uElseIf{\texttt{PAGAM\_PREL\_STD} queue not empty}{
		\textit{processes the first ticket of that type}\;
	}
	\uElse{
		\textit{do nothing}\;
	}
}
\caption{Algoritmo di schedulazione del servente generico}
\label{alg:modello-specifiche-1}
\end{algorithm}

\begin{algorithm}
\SetAlgoLined
\While{true}{
	\uIf{\texttt{SPED\_RIT\_BP} queue not empty}{
		\textit{processes the first ticket of that type}\;
	}
	\uElseIf{\texttt{SPED\_RIT\_STD} queue not empty}{
		\textit{processes the first ticket of that type}\;
	}
	\uElseIf{\texttt{UNICA\_OP\_BP} queue not empty}{
		\textit{processes the first ticket of that type}\;
	}
	\uElseIf{\texttt{PAGAM\_PREL\_BP} queue not empty}{
		\textit{processes the first ticket of that type}\;
	}
	\uElseIf{\texttt{UNICA\_OP\_STD} queue not empty}{
		\textit{processes the first ticket of that type}\;
	}
	\uElseIf{\texttt{PAGAM\_PREL\_STD} queue not empty}{
		\textit{processes the first ticket of that type}\;
	}
	\uElse{
		\textit{do nothing}\;	
	}
}
\caption{Algoritmo di schedulazione del servente dedicato}
\label{alg:modello-specifiche-2}
\end{algorithm}

\newpage
Per ciascun servente generale, ovvero $\forall\ r \in \lbrace 1, 2, \dots, \ded-1, \ded+1, \dots, M \rbrace$, il tempo medio di servizio è pari a:
\begin{equation}
\label{eqn:modello-specifiche-7}
E[S_r] = \frac{p_{UO}}{p_{UO} + p_{PP}} \cdot E[S_{r, UO}] +  \frac{p_{PP}}{p_{UO} + p_{PP}} \cdot E[S_{r, PP}]
\end{equation}
dove il fattore $\frac{1}{p_{UO} + p_{PP}}$ è necessario per normalizzare le probabilità.

Per il servente dedicato si ha:
\begin{equation}
\label{eqn:modello-specifiche-8}
E[S_\ded] = \pi \cdot E[S_{\ded, SR}] + (1-\pi) \cdot E[S_r]
\end{equation}
Tuttavia, è complesso individuare un'espressione in forma chiusa di $\pi$ perché pari alla probabilità che vi siano clienti in possesso di ticket \sr{} da servire, ma che non vi siano richieste \uo{} o \pp{} in servizio su tale sportello.

Assumendo che:
\begin{itemize}
\item Per ogni servente generale:
\begin{equation}
\label{eqn:modello-specifiche-9}
E[S_r] = 15\ min
\end{equation}
con $r \in \lbrace 1, 2, \dots, \ded-1, \ded+1, \dots, M\rbrace$
\item Il tempo medio di servizio per una richiesta di tipo \pp{} sia pari a:
\begin{equation}
E[S_{r, PP}] = 1.5 \cdot E[S_{r, UO}]
\end{equation}
con $r \in \lbrace 1, 2, \dots, M\rbrace$
\item Il tempo medio di servizio per una richiesta di tipo \sr{} sia pari a:
\begin{equation}
\label{eqn:modello-specifiche-11}
E[S_{\ded, SR}] = E[S_r] = 15\ min
\end{equation}
in accordo alla \ref{eqn:modello-specifiche-9}
\end{itemize}
dalla \ref{eqn:modello-specifiche-7} è possibile ricavare:
\begin{equation}
\label{eqn:modello-specifiche-12}
\begin{cases}
E[S_{r,UO}] = \frac{E[S_r]}{(\frac{p_{UO}}{p_{UO} + p_{PP}} + 1.5\cdot \frac{p_{PP}}{p_{UO} + p_{PP}})} = \frac{510}{41} \simeq 12.439024\ min \\[2em]
E[S_{r,PP}] = 1.5\cdot E[S_{r,UO}] = \frac{765}{41} \simeq 18.6585366\ min
\end{cases}
\end{equation}

I risultati ottenuti possono essere verificati con il seguente consistency check:
\begin{equation}
\frac{p_{UO}}{p_{UO} + p_{PP}} \cdot E[S_{r, UO}] +  \frac{p_{PP}}{p_{UO} + p_{PP}} \cdot E[S_{r, PP}] = 15\ min = E[S_r]\qquad \text{\color{forestgreen}\textbf{OK} \checkmark}
\end{equation}

Inoltre, dalla \ref{eqn:modello-specifiche-11} è possibile ricavare la seguente espressione numerica per la \ref{eqn:modello-specifiche-8}:
\begin{equation}
E[S_\ded] = \pi \cdot E[S_{\ded, SR}] + (1-\pi) \cdot E[S_i] = (\cancel{\pi} + (1 - \cancel{\pi}))\cdot E[S_i] = 15\ min
\end{equation}

\begin{table}[ht]
\centering
{\tablecolors
\begin{tabular}{| l | r |}
\hline
Grandezza & Valore misurato nel 2018 \\
\hline
N$^o$ di clienti al giorno & 1.5 mln \\
\hline
N$^o$ di uffici postali & 12 812 \\
\hline
\end{tabular}}
\caption{Estratto dei dati di interesse a partire dalla fonte citata}
\label{table:modello-specifiche-3}
\end{table}

Per ricavare il throughput del sistema si è fatto uso dei dati dati provenienti da \textsl{"Principali dati economici e finanziari di Poste Italiane"}\footnote{\url{https://www.posteitaliane.it/it/performance-finanziaria.html}} nel modo seguente:
\begin{equation}
X = \frac{\# \text{clienti al giorno}}{\#\text{uffici postali} \cdot minuti\_lavoro}_{g} = \frac{1500000}{12812\cdot 480} = \frac{3125}{12812} \simeq 0.243912\ req/min
\end{equation} 
dove $minuti\_lavoro_{g}$ rappresenta l'ammontare medio di minuti che costituiscono la giornata lavorativa e viene computato\footnote{In assenza di dati ufficiali si è assunta una giornata tipo di lavoro per un'attività commericiale.} come segue:
\begin{equation}
\label{eqn:modello-specifiche-15}
minuti\_lavoro_{g} = 8\ h/gg = (60 \cdot 8)\ min/gg = 480\ min/gg
\end{equation}

È opportuno osservare che, sotto l'ipotesi di \textsl{job flow balance}, la frequenza d'arrivo media al centro $\lambda$ coincide con il throughput $X$.

