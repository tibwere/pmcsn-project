\chapter{Modello Delle Specifiche}\label{chp:modello-specifiche}
Le coppie di pedici $(h,j)$ appartenenti al prodotto cartesiano 
\begin{equation}
R = \lbrace \mathtt{BANCO\_POSTA},\ \mathtt{STANDARD}\rbrace \times \lbrace \mathtt{UNICA\_OP},\mathtt{PAGAM\_PREL},\ \mathtt{SPED\_RIT} \rbrace
\end{equation}
vengono mappate su valori numerici, in accordo alla tabella \ref{table:modello-specifiche-1}.
\begin{table}[ht]
\centering
{\tablecolors
\begin{tabular}{| l | l | r |}
\hline
$h$ & $j$ & Indice numerico\\
\hline
\rowcolor{airforceblue!50}
\multicolumn{2}{| c |}{$\mathtt{IDLE}$} & $-1$\\
\hline
$\mathtt{BANCO\_POSTA}$ & $\mathtt{UNICA\_OP}$ & $0$\\
\hline
$\mathtt{BANCO\_POSTA}$ & $\mathtt{PAGAM\_PREL}$ & $1$\\
\hline
$\mathtt{STANDARD}$ & $\mathtt{UNICA\_OP}$ & $2$\\
\hline
$\mathtt{STANDARD}$ & $\mathtt{PAGAM\_PREL}$ & $3$\\
\hline
$\mathtt{BANCO\_POSTA}$ & $\mathtt{SPED\_RIT}$ & $4$\\
\hline
$\mathtt{STANDARD}$ & $\mathtt{SPED\_RIT}$ & $5$\\
\hline
\end{tabular}}
\caption{Mapping tra il modello concettuale e quello delle specifiche}
\label{table:modello-specifiche-1}
\end{table}

Le seguenti variabili, definite:
\begin{itemize}
\item Per ogni istante di tempo $t$
\item Per ogni indice $i$ appartenente alla terza colonna della tabella \ref{table:modello-specifiche-1}
\end{itemize}
identificano univocamente una rappresentazione a livello delle specifiche dello stato del sistema.
\begin{itemize}
\item Per ogni $i \neq -1$, il numero di clienti presenti nel sistema è dato da:
\begin{equation}
Customers_i(t) \in \mathbb{N}
\end{equation}
\item Il numero di clienti in servizio allo sportello $r$-esimo (con $r \in \lbrace 1, 2, \dots, \ded-1, \ded+1, \dots, M \rbrace$) è dato da:
\begin{equation}
\label{eqn:modello-specifiche-3}
Server_r(t) \in \lbrace -1, 0, \dots, 3 \rbrace
\end{equation}

in accordo alla tabella \ref{table:modello-specifiche-1} (e.g. se il servente $r$-esimo sta elaborando un ticket di tipo \uo{} \textsl{BancoPosta} all'istante $t$, allora $Server_r(t)=0$).
\item Il numero di clienti in servizio allo sportello dedicato è dato da:
\begin{equation}
\label{eqn:modello-specifiche-4}
Server_\ded(t) \in \lbrace -1, 0, \dots, 5 \rbrace
\end{equation}

in accordo alla tabella \ref{table:modello-specifiche-1}\footnote{La variabile del server $k$-esimo può assumere un insieme più vasto di valori perché è l'unico che può elaborare ticket di tipo \sr{}.}.
\end{itemize}

Dalle variabili appena descritte, è immediato ricavare il numero di richieste in coda per ciascuna tipologia di ticket:
\begin{equation}
Queue_i(t) = Customers_i(t) - \# \lbrace Server_v(t)\ \vert\ Server_v(t) = i \rbrace
\end{equation}
con $v \in \lbrace 1, 2, \dots, M \rbrace$ (ovvero compreso anche il server dedicato).

Infine, per ricavare il numero di clienti totale nel sistema è possibile sfruttare la seguente relazione:
\begin{equation}
Customers(t) = \sum_{i=0}^{5} Customers_i(t)
\end{equation}

Di seguito sono riportate alcune assunzioni che saranno alla base di questa e delle successive fasi dello studio:
\begin{itemize}
\item I clienti arrivano all'ufficio postale ad istanti di tempo casuali, il che implica:
\begin{itemize}
\item Distribuzione poissoniana degli arrivi.
\item Distribuzione esponenziale dei tempi di interarrivo.
\end{itemize}
\item La probabilità che un cliente sia titolare di un conto \textsl{BancoPosta} è pari a $p_{BP} = 0.25$.
\item Le probabilità con cui ciascuna tipologia di ticket viene acquisita sono le seguenti:
\begin{equation*}
\begin{array}{l c l}
\uo{} & \rightarrow & p_{UO} = 0.5 \\
\pp{} & \rightarrow & p_{PP} = 0.35 \\
\sr{} & \rightarrow & p_{SR} = 0.15
\end{array}
\end{equation*} 
\item I tempi di servizio sono distribuiti esponenzialmente.
\item I clienti afferenti ad una stessa coda vengono serviti in accordo ad una disciplina FIFO (First-In, First-Out).
\item Il servizio di un cliente non può essere interrotto per favorire l'avanzamento di un altro con priorità superiore.
\end{itemize}

La politica di scheduling del sistema assume caratteristiche tipiche sia dello scheduling size-based che di quello astratto. In particolare:
\begin{itemize}
\item Ha caratteristiche astratte perché la probabilità di ricadere in una delle classi di priorità è funzione della frequenza con cui l'utente prende quella tipologia di ticket e non del tempo necessario all'elaborazione della richiesta da parte del servente. Infatti, il tempo necessario ad elaborare un ticket della classe $i$ non è necessariamente compreso in un intervallo $(x_{i-1}, x_i]$ bensì può assumere valori su tutto il semiasse reale positivo. 
\item Ha caratteristiche size-based perché classi di priorità diverse non condividono lo stesso tempo di servizio medio, ovvero $\exists\ i,j\ t.c.\ E[S_i] \neq E[S_j]$.
\end{itemize}

Al fine di agevolare la comprensione del funzionamento dello scheduler di sistema, di seguito sono riportati gli pseudocodici \ref{alg:modello-specifiche-1} e \ref{alg:modello-specifiche-2} che descrivono, rispettivamente, il comportamento del servente generico e di quello dedicato.

\begin{algorithm}
\caption{Algoritmo di schedulazione del servente generico}
\label{alg:modello-specifiche-1}
\begin{algorithmic}[1]
\Procedure{GeneralPurposeServer}{}
\While{true}
	\If{customer owns \textsl{BancoPosta}}
		\If{\textsl{UnicaOperazioneBP} queue not empty}
			\State \textit{processes the first ticket of that type}
		\Else
			\If{\textsl{PagamentiPrelieviBP} queue not empty}
				\State \textit{processes the first ticket of that type}
			\EndIf
		\EndIf
	\Else
		\If{\textsl{UnicaOperazione} queue not empty}
			\State \textit{processes the first ticket of that type}
		\Else
			\If{\textsl{PagamentiPrelievi} queue not empty}
				\State \textit{processes the first ticket of that type}
			\EndIf
		\EndIf
	\EndIf
\EndWhile
\EndProcedure
\end{algorithmic}
\end{algorithm}

\begin{algorithm}
\caption{Algoritmo di schedulazione del servente dedicato a ticket \sr{}}
\label{alg:modello-specifiche-2}
\begin{algorithmic}[1]
\Procedure{DedicatedServer}{}
\While{true}
	\If{customer owns \textsl{BancoPosta}}
		\If{\textsl{SpedizioneRitiriBP} queue not empty}
			\State \textit{processes the first ticket of that type}
		\Else
			\If{\textsl{UnicaOperazioneBP} queue not empty}
				\State \textit{processes the first ticket of that type}
			\Else
				\If{\textsl{PagamentiPrelieviBP} queue not empty}
					\State \textit{processes the first ticket of that type}
				\EndIf
			\EndIf
		\EndIf
	\Else
		\If{\textsl{SpedizioneRitiri} queue not empty}
			\State \textit{processes the first ticket of that type}
		\Else
			\If{\textsl{UnicaOperazione} queue not empty}
				\State \textit{processes the first ticket of that type}
			\Else
				\If{\textsl{PagamentiPrelievi} queue not empty}
					\State \textit{processes the first ticket of that type}
				\EndIf
			\EndIf
		\EndIf
	\EndIf
\EndWhile
\EndProcedure
\end{algorithmic}
\end{algorithm}

Per ciascun servente generale, ovvero $\forall\ i \neq \ded$, il tempo medio di servizio è pari a:
\begin{equation}
\label{eqn:modello-specifiche-7}
E[S_i] = \frac{p_{UO}}{p_{UO} + p_{PP}} \cdot E[S_{i, UO}] +  \frac{p_{PP}}{p_{UO} + p_{PP}} \cdot E[S_{i, PP}]
\end{equation}

Per il servente dedicato si ha:
\begin{equation}
E[S_\ded] = \pi \cdot E[S_{\ded, SR}] + (1-\pi) \cdot E[S_i]
\end{equation}
Tuttavia, è complesso individuare un'espressione in forma chiusa di $\pi$ perché pari alla probabilità che vi siano clienti in possesso di ticket \sr{} da servire, ma che non vi siano richieste \uo{} o \pp{} in servizio su tale sportello.

\newpage

Assumendo che:
\begin{itemize}
\item Per ogni servente generale:
\begin{equation}
E[S_i] = 10\ min
\end{equation}
con $i \in \lbrace 1, 2, \dots, \ded-1, \ded+1, \dots, M\rbrace$.
\item Il tempo medio di servizio per una richiesta di tipo \pp{} sia pari a:
\begin{equation}
E[S_{i, PP}] = 1.5 \cdot E[S_{i, UO}]
\end{equation}
con $i \in \lbrace 1, 2, \dots, M\rbrace$.
\item Il tempo medio di servizio per una richiesta di tipo \sr{} sia pari a:
\begin{equation}
E[S_{\ded, SR}] = 2 \cdot E[S_{i, UO}]
\end{equation}
con $i \in \lbrace 1, 2, \dots, M\rbrace$.
\end{itemize}
dalla \ref{eqn:modello-specifiche-7} è possibile ricavare:
\begin{equation}
\begin{cases}
E[S_{i,UO}] = \frac{E[S_i]}{(\frac{p_{UO}}{p_{UO} + p_{PP}} + 1.5\cdot \frac{p_{PP}}{p_{UO} + p_{PP}})} = \frac{340}{41} \simeq 8.2926829\ min \\[2em]
E[S_{i,PP}] = 1.5\cdot E[S_{i,UO}] = \frac{510}{41} \simeq 12.4390244\ min \\[2em]
E[S_{\ded,SR}] = 2 \cdot E[S_{i,UO}] = \frac{680}{41} \simeq 16.5853658\ min
\end{cases}
\end{equation}

I risultati ottenuti possono essere verificati con il seguente consistency check:
\begin{equation}
\frac{p_{UO}}{p_{UO} + p_{PP}} \cdot E[S_{i, UO}] +  \frac{p_{PP}}{p_{UO} + p_{PP}} \cdot E[S_{i, PP}] = 10\ min = E[S_i]\qquad \text{\color{forestgreen}\textbf{OK} \checkmark}
\end{equation}

\begin{table}[ht]
\centering
{\tablecolors
\begin{tabular}{| l | r |}
\hline
Grandezza & Valore misurato nel 2019 \\
\hline
N$^o$ di clienti al giorno & 1.4 mln \\
\hline
N$^o$ di uffici postali & 12 809 \\
\hline
\end{tabular}}
\caption{Estratto dei dati di interesse a partire dalla fonte citata}
\label{table:modello-specifiche-2}
\end{table}

Per ricavare il throughput del sistema $X$ si è fatto uso dei dati dati provenienti da:
\begin{itemize}
\item \textsl{"Principali dati economici e finanziari di Poste Italiane"}\footnote{\url{https://www.posteitaliane.it/it/performance-finanziaria.html}}
\item Orari di apertura di un ufficio postale di Fiumicino\footnote{\url{https://www.centroaperture.it/orari-apertura/fiumicino-poste-italiane-112743521}}
\end{itemize}
nel modo seguente:
\begin{equation}
X = \frac{\# \lbrace \text{clienti al giorno} \rbrace}{\# \lbrace \text{uffici postali} \rbrace} \cdot \frac{1}{\frac{60}{7} \cdot 60} = \frac{24500}{115281} \simeq 0.21252418\ req/min
\end{equation} 
dove le ore di lavoro giornaliere sono state calcolate come segue:
\begin{equation}
\label{eqn:modello-specifiche-15}
\frac{11h \cdot 5gg + 5h \cdot 1gg + 0h \cdot 1gg}{7gg} = \frac{60}{7} \simeq 8.57142856\ h/gg  
\end{equation}

È opportuno osservare che, sotto l'ipotesi di \textsl{job flow balance}, la frequenza d'arrivo media al centro $\lambda$ coincide con il throughput $X$.


















