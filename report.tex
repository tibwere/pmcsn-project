\documentclass{article}

% Package necessari
\usepackage[a4paper]{geometry}
\usepackage[utf8]{inputenc}
\usepackage[italian]{babel}
\usepackage[T1]{fontenc}
\usepackage{amsmath}
\usepackage{amssymb}
\usepackage{graphicx}
\usepackage[table, dvipsnames]{xcolor}
\usepackage{listings}
\usepackage{cancel}
\usepackage{hyperref}
\usepackage{enumitem}

% Impostazione delle lunghezze di alcuni elementi del documento
\setlength{\parskip}{1em}
\setlength{\parindent}{0em}
\setlength{\arrayrulewidth}{0.1em}

% Informazioni per la title page
\title{\small{Corso di Performance Modeling of Computer Systems \& Networks} \\
\huge{Studio delle prestazioni di un Ufficio Postale di \\
\colorbox{yellow}{\color{blue} \textbf{Poste} Italiane}}}
\date{A.A. 2020/2021}
\author{A. Chillotti 
\and 
C. Cuffaro 
\and 
S. Tiberi}

% Impostazione del package hyperref
\hypersetup{
    colorlinks=true,
    linktocpage=true,
    linkcolor=blue,
    urlcolor=blue,
    pdftitle={Studio delle prestazioni di un Ufficio Postale di Poste Italiane},
    pdfauthor={A. Chillotti, C. Cuffaro e S. Tiberi},
}

% Colori per i listing
\definecolor{code_red}{rgb}{0.6,0,0} % strings
\definecolor{code_green}{rgb}{0.25,0.5,0.35} % comments
\definecolor{code_purple}{rgb}{0.5,0,0.35} % keywords
\definecolor{code_background}{rgb}{0.95,0.95,0.92} % background
 
% Stile del codice standard (C)
\lstset{
	language=C, 
	backgroundcolor=\color{code_background},
	frame=single,
	basicstyle=\ttfamily,
	keywordstyle=\color{code_purple}\bfseries,
	stringstyle=\color{code_red},
	commentstyle=\color{code_green},
	numbers=left,
	numberstyle=\small\color{gray},
	numbersep=5pt,
	tabsize=4,
	showtabs=false,
	showspaces=false,
	showstringspaces=false,
	escapechar=|, 
	captionpos=b,
	breaklines=true,
}

% Spaziatura tabelle
\renewcommand{\arraystretch}{1.5}

\graphicspath{ {./figs/} }
% Definizione del colore delle tabelle
\newcommand{\tablecolors}[1][2]{\rowcolors{#1}{gray!50}{gray!25}}

% Definizione dello stile da usare per la P di probabilità (grassetto in math-mode)
\newcommand{\pr}{\boldsymbol{P}}

% Forzatura del displaystyle in math-mode
\everymath\expandafter{\the\everymath\displaystyle}

\newcommand{\scaption}[1]{\caption{\small{#1}}}

\begin{document}
\maketitle
\tableofcontents

\section{Presentazione del caso di studio}
Il sistema oggetto dell'analisi in questione eroga le seguenti tipologie di servizi:
\begin{enumerate}
\item \textbf{Unica Operazione} (e.g. ricarica \textsl{PostePay}, invio raccomandata e pagamento di massimo 3 bollettini)
\item \textbf{Pagamenti \& Prelievi} (e.g. pagamento di un numero arbitrario di bollettini, bollo auto e libretti)  
\item \textbf{Spedizioni \& Ritiri} (e.g. invio corrispondenza, lettere, pacchi e raccomandate)
\end{enumerate}

Per essere serviti i clienti possono:
\begin{itemize}
\item Recarsi all'ufficio postale, prendere un ticket relativo al servizio a cui sono interessati e mettersi in coda in attesa del proprio turno. Nel caso in cui essi dimostrano di essere titolari di un conto \textsl{BancoPosta} potranno accodarsi in una fila dedicata.
\item Prenotare un ticket mediante l'applicazione \textsl{"Ufficio Postale"} per una determinata fascia oraria, al fine di essere serviti dal primo sportello disponibile entro 40 minuti, ma non prima, dall'orario di prenotazione.
\end{itemize}

Un insieme di sportelli serve le richieste degli utenti in accordo alle seguenti regole: 
\begin{enumerate}[label=R\arabic*)]
\item I clienti titolari di un conto \textsl{BancoPosta} vengono serviti con una priorità maggiore rispetto agli altri, indipendentemente dal ticket scelto.
\item Poiché, per definizione, ticket di tipo \textbf{Unica Operazione} dovrebbero richiedere meno tempo per essere processati, viene assegnata loro la massima priorità
\item I ticket di tipo \textbf{Spedizioni \& Ritiri} vengono serviti da uno sportello dedicato il quale, in assenza di questa tipologia di ticket, opera come gli altri. Il comportamento di tale servente è schematizzato in figura \ref{fig:presentazione-1}. 
\end{enumerate}

\begin{figure}[ht]
\centering
\includegraphics[width=0.75\linewidth]{presentazione-1}
\scaption{Schema del comportamento del servente dedicato ai ticket di tipo \textbf{Spedizioni \& Ritiri}}
\label{fig:presentazione-1}
\end{figure}
\chapter{Obiettivi dello studio}\label{chp:obiettivi}
L'obiettivo dello studio è quello di minimizzare il numero degli sportelli operativi in un'intera giornata lavorativa, al fine di garantire il soddisfacimento di differenti requisiti di qualità per ciascuna tipologia di servizio illustrata nella presentazione del caso di studio (cap. \ref{chp:presentazione}):

\begin{itemize}
\item I clienti titolari di un conto \textsl{BancoPosta}:
\begin{enumerate}[label=QoS-\arabic*), align=left, leftmargin=*]
\item Possessori di ticket di tipo \uo{}, devono sperimentare un tempo d'attesa non superiore a $15\ min$.
\item Possessori di ticket di tipo \pp{}, devono sperimentare un tempo d'attesa non superiore a $20\ min$.
\item Possessori di ticket di tipo \sr{}, devono sperimentare un tempo d'attesa non superiore a $20\ min$.
\end{enumerate}
\item I clienti \textbf{non} titolari di un conto \textsl{BancoPosta}:
\begin{enumerate}[label=QoS-\arabic*), align=left, leftmargin=*]
\setcounter{enumi}{3}
\item Possessori di ticket di tipo \uo{}, devono sperimentare un tempo d'attesa non superiore a $30\ min$.
\item Possessori di ticket di tipo \pp{}, devono sperimentare un tempo d'attesa non superiore a $90\ min$.
\item Possessori di ticket di tipo \sr{}, devono sperimentare un tempo d'attesa non superiore a $45\ min$.
\end{enumerate}
\end{itemize}

\end{document}